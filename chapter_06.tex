\chapter{Qmail}\index{Qmail} % kapt 5

Før du leser dette kapitlet så er det lurt å ha sett eller i det minst bladd gjennom
forelesningen om Qmail. Jeg gjentar ikke alt som står i forelesningen her.

I læreboka kapittel 5.10.2-5.10.9 omtales (manuell) installasjon og konfigurering av 
\prog{qmail}, \prog{daemontools} og \prog{ucspi-tcp}. Dette kan dere se helt bortifra.
Vi skal selvfølgelig bruke Apt.

Testing av Qmail, som utføres i 5.10.6, er gjengitt her i kompendiet. Jeg anbefaler
at disse testene utføres. Jeg anbefaler også at dere tester å \textquote{snakke}
SMTP ved hjelp av \prog{telnet} - se 5.10.9 i læreboka eller opptak fra forelesning.

Ellers kan \href{http://www.lifewithqmail.org/lwq.html}{Life with Qmail} være en god ressurs
for å lære om Qmail (hvis man ikke har læreboka).

\begin{remark}
Aliaser og alias-brukerens spesielle roller blir omtalt i neste kapittel.
\end{remark}


\section{Installasjon}

På grunn av en (fersk) feil i pakken \prog{qmail-run} så må vi installere Daemontools 
først:

\cmd{apt install daemontools-run}

Etterpå kan du installere Qmail. Vi installerer pakken \prog{qmail-run} slik at 
Qmail automatisk bruker Daemontools.

\cmd{apt install qmail-run}

Etter dette kjører Qmail - din server kan sende og motta epost. Du kan ikke bruke 
den vanlige \prog{service}-kommandoen for Qmail siden den er kontrollert av 
Daemontools. I tillegg til Daemontools-kommandoene så fins det \prog{qmailctl}.
Hvis du lurer på om Qmail faktisk kjører, bruk

\cmd{qmailctl stat}\index{qmailctl|see{Qmail}}\index{Qmail!qmailctl}

Hvis du skal lese epost lokalt på serveren, så anbefaler jeg programmet \prog{mutt}. 
Dette minner ikke mye om Outlook eller G-mail, men du klarer nok å lese og sende epost.
Du installerer det som vanlig med

\cmd{apt install mutt}

\section{Konfigurasjon}

\subsection{Daemontools}\index{Daemontools}

\begin{remark}
Service-mappa i Debian er \file{/etc/service} i motsetning til
læreboka som bruker \file{/service}. Prøv å huske det ...
\end{remark}

Daemontools er en samling verktøy for å administrere tjenester (services).
I motsetning til Systemd, så overvåker Daemontools tjenestene og starter 
dem på nytt hvis de dør. Dette er meget praktisk.

For å administrere en tjeneste oppretter man en mappe i \file{/etc/service}
som inneholder en fil \file{run} som starter tjenesten. Resten er automagisk.

Pakken \prog{qmail-run} oppretter \file{/etc/service/qmail-smtpd} (og to andre mapper)
som gjør at Qmail blir kontrollert Daemontools.

Daemontools har egentlig bare to kommandoer - \prog{svstat} som gir status for en tjeneste
og \prog{svc} som kontrollerer tjenesten.
\index{svstat|see{Daemontools}}\index{Daemontools!svstat}
\index{svc|see{Daemontools}}\index{Daemontools!svc}

Med \prog{qmail-smtpd} som et eksempel har du følgende nesten selvforklarende eksempler:

\cmd{svstat /etc/service/qmail-smtpd}\\
\cmd{svc -u /etc/service/qmail-smtpd}\\
\cmd{svc -p /etc/service/qmail-smtpd}

Hvis du lurer på hva de to siste eksemplene gjør, les man-siden:

\cmd[\$]{man svc}

\subsection{Kontrollfiler}

Alle datafilene til Qmail ligger i \file{/var/lib/qmail/}. Legg merke til 
at hjemmeområdet til \file{alias}-brukeren er i samme mappe - \file{/var/lib/qmail/alias/}.

Kontrollfilene er tilgjengelig i \file{/etc/qmail}. (\file{/var/lib/qmail/control} er en symbolsk
lenke til \file{/etc/qmail}.) 

De viktigste filene er \file{me}, \file{locals} og \file{rcpthosts}. Du finner informasjon 
om kontrollfilene i man-sidene for \prog{qmail-smtpd} og \prog{qmail-inject}. For en komplettliste
av kontrollfiler kjør:

\cmd[\$]{man qmail-control}

\section{Testing}

I de følgende testene/eksemplene skal du bytte ut \file{minbruker} med ditt brukernavn.
Målet med testene er å skjønne hvordan Qmail fungerer. Du sender selvfølgelig vanligvis
epost via en MUA/epostklient (som \prog{mutt}) - ikke via \prog{qmail}-kommandoer.

\begin{remark}
Standard innboks for en bruker i Debian er \file{/var/mail/brukernavn} i motsetning
til læreboka som bruker \file{/home/brukernavn/Mailbox}.

\end{remark}

Loggfilen for alle testene er \file{/var/log/qmail/current}. Hvis du ønsker å se filen
med forståelig tidsstempel, bruk

\cmd{cat /var/log/qmail/current | tai64nlocal}

Ofte ønsker du bare å se de f.eks. 7 siste linjene i loggfilen:

\cmd{cat /var/log/qmail/current | tail -n 7 | tai64nlocal}

NB! Loggfilen for innkommende (ekstern) epost er \file{/var/log/qmail/smtpd/current}

\subsection*{Testmelding til lokal adresse}

Sender en melding som kun har et \textquote{to}-felt og ingen kropp.

\cmd[\$]{echo \textquotedbl{}to: minbruker\textquotedbl{} | qmail-inject}

Sjekk loggen og brukerens innboks (enten med \prog{nano} eller \prog{mutt}).

\subsection*{Testmelding til ukjent lokal adresse}

\cmd[\$]{echo \textquotedbl{}to: brukersomikkefins\textquotedbl{} | qmail-inject}

Sjekk loggen og brukerens innboks (enten med \prog{nano} eller \prog{mutt}).
Legg merke til Bounce-meldingen siden adressen ikke fantes.

\subsection*{Testmelding til postmaster}

\cmd[\$]{echo \textquotedbl{}to: postmaster\textquotedbl{} | qmail-inject}

Sjekk loggen. Hvor havner eposten?

\subsection*{Testmelding til ekstern adresse}

Bruk din vanlige epostadresse i denne testen.

\cmd[\$]{echo \textquotedbl{}to: deg@etellerannetdomene\textquotedbl{} | qmail-inject}

Sjekk loggen.

\subsection*{Testmelding til ukjent avsender og ukjent mottaker}

\cmd[\$]{echo \textquotedbl{}to: ukjentmottaker\textquotedbl{} | qmail-inject -f ukjentavsender}

Sjekk loggen. Legg merke til dobbel Bounce-meldingen siden også avsenderadressen 
ikke fantes.
