\chapter{ezmlm} % kapt 6

\section{Installasjon}

Hvis du går rett på sak, så forteller

\cmd{apt install ezmlm}

deg at det ikke fins en pakke \textquote{ezmlm}. Hvis du prøver å søke
etter en pakke som heter noe med \textquote{ezmlm}, så finner du heller ikke noe relevant:

\cmd{apt search ezmlm}

Du bør nå sjekke hele Debian sitt pakkearkiv i nettleseren. Bruk
\url{https://packages.debian.org/pakkenavn}
eller i dette tilfellet \url{https://packages.debian.org/ezmlm} 

Du ser at det fins en pakke \file{ezmlm-idx} men at den ikke er tilgjengelig 
for den stabile utgaven av Debian som vi bruker. Du kan sjekke hvilken utgave 
av Debian du henter pakker fra, ved å se i filen \file{/etc/apt/sources.list}. 
(Den nåværende stabile utgaven av Debian heter \textquote{stretch}.)
\index{APT!sources.list}

For installere \file{ezmlm-idx} så kan vi enten finne .deb-filen via søket på
\url{https://packages.debian.org/} og installere .deb-filen manuelt, 
eller oppdatere \file{sources.list} med experimental-utgaven. Vi går for det siste.

På slutten av \file{/etc/apt/sources.list} legg til

\begin{filedata}
# To enable install of ezmlm and other packages (from experimental)
deb http://ftp.uio.no/debian/ experimental main
deb-src http://ftp.uio.no/debian/ experimental main
\end{filedata}

Etterpå oppdaterer du indeksen til Apt og installerer som vanlig.

\begin{remark}
Å legge til flere utgaver i \file{/etc/apt/sources.list} er potensielt farlig siden
du ikke lenger har kontroll med hvilken utgave av Debian pakkene hentes fra.
Les mer om dette i 
\href{https://debian-handbook.info/browse/stable/sect.apt-get.html\#sect.apt-upgrade}%
{The Debian Administrator's Handbook} (etter den grå boksen
\textquote{Incremental upgrade}).
\end{remark}

\iffalse

\subsection{Manuell installasjon av ezmlm}

For å finne filen gikk jeg til \url{https://packages.debian.org/ezmlm-idx} og havnet
til slutt på \url{https://packages.debian.org/experimental/i386/ezmlm-idx/download}. 
Selve installasjonen blir da:

\cmd{wget http://ftp.no.debian.org/debian/pool/main/e/ezmlm-idx/ezmlm-idx_7.1.1-1~exp0+b1_i386.deb}
\cmd{dpkg -i ezmlm-idx_7.1.1-1~exp0+b1_i386.deb}

\fi


\section{Konfigurasjon}
\index{ezmlm-make}

Du oppretter en liste med \prog{ezmlm-make}. Hvis du oppretter en liste for 
alias-brukeren, så blir det en sentral liste. Listeadressen kan f.eks. 
være \url{hvasomhelst@lpcXXX.himolde.no}. For å opprette slike 
lister må du være \file{root}. Hvis en vanlig bruker skal opprette en liste, 
så blir den knyttet til brukernavnet via alias-mekanismen og listeadressen blir 
\url{brukenavn-hvasomhelst@lpcXXX.himolde.no}.

\begin{remark}
For å unngå problemer med rettigheter/eier når du oppretter en sentral
liste er det lurt å passe på at du er alias-brukeren f.eks. med sudo:

\cmd{sudo -u alias ezmlm-make \textasciitilde{}alias/mappe [...]}
\end{remark}

Man vedlikeholder lister med de forskjellig \prog{ezmlm-CMD}-programmene - 
f.eks. \prog{ezmlm-sub}, \prog{ezmlm-unsub}.
Mange av disse oppgavene kan også utføres ved å sende en epost til
en spesiell adresse.
