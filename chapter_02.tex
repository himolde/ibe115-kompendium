\chapter{Programvareinstallasjon}\index{pakkebehandling}

En av de største fordelene med Linux og andre FOSS operativsystem er at de
har fremragende støtte for installasjon og vedlikehold av programvare. 
All installasjon av programvare gjøres med såkalte pakkebehandlere. Man laster ikke
ned programvare direkte fra (hundrevis av) nettsteder med de farene det medfører.
All programvare er ferdig pakket og blir hentet fra et sentralt arkiv. 

En videre diskusjon om fordelene med pakkebehandlersystem er utelatt. Jeg har heller 
ikke tatt med informasjon om manuell kompilering (bygging) og installasjon av programvare,
og de åpenbare problemene med dette. (Læreboka har noen eksempler på kompilering med
klassikske verktøy som \prog{configure} og \prog{make}.)

Det fins flere forskjellige pakkebehandlersystem for Linux hvor kanskje dpkg og 
RPM er de mest kjente. Disse systemene kan benyttes via forskjellige pakkebehandlere
hvor APT er vanligst for dpkg og kanskje YUM er mest kjent for RPM. (Som vanlig med 
Linux er det utrolig mange forskjellige muligheter.)

Siden vi bruker Debian, så skal vi altså bruke APT\index{APT}. 
Du kan bruke APT via kommandoene \prog{apt}\index{APT!apt},
\prog{apt-get}\index{APT!apt-get} (med vennen \prog{apt-cache}) 
eller \prog{aptitude}\index{APT!aptitude}. Det bør nevnes at 
\prog{aptitude} også er et nesten grafisk (såkalt ncurses-basert) grensesnitt til APT.
I kurset vil jeg bruke \prog{apt} når jeg ikke glemmer meg og bruker \prog{apt-get}.
Jeg håper vi slipper å bruke \prog{dpkg}-kommandoene som er på et lavere nivå - mindre 
brukervennlig og større sjanse for å ødelegge.

Dette kompendiet vil ikke gi noen introduksjon til APT utover noen få eksempler nedenfor.
For mer informasjon om bruken av APT les for eksempel kapittel 6.2 i 
\href{https://debian-handbook.info/browse/stable/}{The Debian Administrator's Handbook} eller 
kapittel 2 i 
\href{https://www.debian.org/doc/manuals/debian-reference/}{Debian Reference} eller 
\href{https://www.debian.org/doc/manuals/refcard/refcard}{Debian sitt referansekort} som 
er veldig kortfattet og beregnet for å skrives ut tosidig på ett A4-ark).

\section{Noen eksempler for APT}\index{APT!apt}

Oppdatere pakkearkivet slik at nye pakker og oppdateringer/nye utgaver blir tilgjengelig:

\cmd{apt update}

Oppgradere alle programmer/pakker du har installert:

\cmd{apt upgrade}

Installere et program/en pakke (med alle avhengigheter):

\cmd{apt install pakkenavn}

Fjerne et program:

\cmd{apt remove pakkenavn} 

eller 

\cmd{apt purge pakkenavn}

om du også vil fjerne konfigurasjonen for pakken.

\section{\prog{apt} VS \prog{apt-get}}

Sitat fra man-siden til \prog{apt}:

\begin{quote}
apt provides a high-level commandline interface for the package management system. 
It is intended as an end user interface and enables some options better suited for 
interactive usage by default compared to more specialized APT tools like 
apt-get(8) and apt-cache(8).
\end{quote}

Med andre ord; \prog{apt} er supert for interaktiv bruk. For skripting bruker man
fortsatt \prog{apt-get}.\index{APT!skripting}
