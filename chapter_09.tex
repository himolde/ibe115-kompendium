\chapter{Bind, tinydns og dnscache} % kapt 8
% #################################

\index{Bind}\index{djbdns}
\index{tinydns|see{djbdns}}\index{djbdns!tinydns}
\index{dnscache|see{djbdns}}\index{djbdns!dnscache}

Et lite tips: Ikke installer Bind før du har gjort deg ferdig med
dnscache og tinydns. Bind, hvis du ikke endrer konfigurasjonen, vil
lytte på alle IP-adresser etter installasjon og komme i konflikt med
dnscache og tinydns.
 
% ####################
\section{Installasjon}
% ####################

Installasjon av Bind fungerer som standard og vil ikke bli omtalt her. 
Du må bare husk at pakken heter \file{bind9}. Vi skal fokuserer på djbdns nedenfor.

Akkurat som for Ezmlm, så fins det ingen pakke for djbdns i den stabile utgaven av Debian som vi bruker.
Dessverre fins det ikke lenger noen pakke for andre utgaver av Debian heller. 
På \url{https://packages.debian.org/djbdns} finner du bare en pakke for arm64-prosessoren. 
Vi må derfor bruke et skittent triks:

Laste ned pakken fra \url{http://snapshot.debian.org/}

Installasjon av pakker fra \url{http://snapshot.debian.org/} fungerer greit fordi 
pakken blir registrert av pakkesystemet, og alle avhengigheter blir behandlet som vanlig. 
Generelt kan man ikke regne med at pakker fra snapshots.debian.org fungerer siden avhengigheter
kan mangle.

Du finner mulige pakker via \url{http://snapshot.debian.org/package/pakkenavn/}, altså 
\url{http://snapshot.debian.org/package/djbdns/}. På den siden vil du etterhvert finne riktig
deb-fil for i386:

\file{http://snapshot.debian.org/archive/debian/20130529T040126Z/pool/main/d/djbdns/djbdns\_1.05-9~exp2\_i386.deb}

Kopier URL-en over og last ned med \prog{wget}. 
Deretter installerer du manuelt for eksempel med:

\cmd{dpkg -i djbdns\_1.05-9~exp2\_i386.deb}

% ######################################################
\section{Konfigurasjon av dnscache-conf og tinydns-conf}
% ######################################################

\index{djbdns!tinydns}
\index{djbdns!dnscache}
\index{dnscache-conf|@see{djbdns}}\index{djbdns!dnscache-conf}
\index{tinydns-conf|@see{djbdns}}\index{djbdns!tinydns-conf}


Vi skal konfigurere dnscache og tinydns fra grunnen av - som beskrevet i læreboka (for en gangs skyld).
Når vi installerte Qmail, slapp vi billig unna. Pakken \file{qmail-run} installerer for eksempel 
\file{qmail-uids-gids} som oppretter brukere. Vi må gjøre dette selv:

\cmd{useradd -s /bin/false Gdnscache}\\
\cmd{useradd -s /bin/false Gtinydns}\\
\cmd{useradd -s /bin/false Gdnslog}

PS! Du kan ikke logge inne som disse brukerne siden de har \prog{/bin/false} som SHELL. 

Siden det ikke fins en \file{djbdns-run} pakke så må vi også opprette run-filer osv. for Daemontools.
Heldigvis fins det verktøy for dette. Vi må bruke \prog{dnscache-conf} og \prog{tinydns-conf}.
Det viktigste med disse kommandoene er å benytte riktig IP-adresser.

For å sette opp dnscache som en lokal tjeneste (standard), bruk

\cmd{dnscache-conf Gdnscache Gdnslog /etc/dnscache}

For å sette opp tinydns som en offentlig tjeneste, bruk

\cmd{tinydns-conf Gtinydns Gdnslog /etc/tinydns 158.38.74.XYZ}

Ja, du må bytte ut \textquote{XYZ} med ditt lpc-nummer.
 
For å starte dnscache eller tinydns som en tjeneste, lenker 
du bare opp disse mappene i \file{/etc/service}.
Du kan sjekke med \prog{svstat} (etter noen sekunder).

\cmd{ln -s /etc/dnscache /etc/service}\\
\cmd{svstat /etc/service/dnscache}

\cmd{ln -s /etc/tinydns /etc/service}\\
\cmd{svstat /etc/service/tinydn}

% ############################################
\subsection{Legge til ressursdata i tinydns-conf}
% ############################################

\index{djbdns!tinydns}

Dette er omtalt i lærebok på side 243 og utover. Jeg gjengir i korte trekk her.

For å legge til ressursdata så må du være i mappa \file{/etc/tinydns/root/}.
Det er i alle fall mye enklere ... Hver gang du har lagt til ressursdata, 
må du kompilere data-filen slik at tinydns ser endringen. Dette gjør du med 

\cmd{make}

Kanskje du må installere \file{make}-pakken?

For å legge til navnetjenere, NS-oppføringer, bruk:

\cmd{./add-ns lpcXYZ.himolde.no Offentlig-IP}\\
\cmd{./add-ns lpcXYZ.himolde.no IP-alias}

For å legge til domenenavn, A-oppføringer, bruk:

\cmd{./add-host lpcXYZ.himolde.no Offentlig-IP}

For å legge til et domenenavn til for samme IP (alias), bruk:

\cmd{./add-alias somename.lpcXYZ.himolde.no Offentlig-IP}

Dette gir ikke en vanlig CNAME-oppføring.

Alle disse \file{add}-kommandoene er egentlig bare wrappere rundt
\prog{tinydns-edit}. Du kan derfor slå opp i man-siden for dette programmet
hvis du lurer på noe.
\index{tinydns-edit|@see{djbdns}}\index{djbdns!tinydns-edit}

% #############################
\section{Konfigurasjon av Bind}
% #############################

\index{Bind}

Dokumentasjonen til gjeldende utgave av Bind finner du alltid på\\
\url{https://www.isc.org/downloads/bind/doc/} \cite{docs:bind}\\
Debian bruker for øyeblikket utgave 9.10 av Bind.

Et klassisk eksempel på sonefile finner du på\\
\url{http://www.zytrax.com/books/dns/ch6/mydomain.html}\\
Alle ressursdata for et domene legges inn i sonefila.

\begin{remark}
Selv om læreboka nevner det, så trenger du altså ikke legge til brukere osv 
som du gjorde for dnscache og tinydns. Du trenger heller ikke opprette mapper eller noe annet.
Bind fungerer ut av boksen.
\end{remark}

Legg spesielt merke til at Bind sin konfigurasjon i Debian ligger i \file{/etc/bind}, 
ikke \file{/etc/bind9} (som ville vært logisk siden pakken heter \file{bind9}) 
eller \file{/etc/named}. 

Kommentarene nedenfor er rettet mot arbeidet med den obligatoriske oppgaven
om DNS. Ikke alt har generell gyldig. Jeg forutsetter også at du har lagt
til et IP-alias for nettverkskortet ditt, slik at du har to offentlige IP-adresser
for serveren din. 

Du bør endre \file{/etc/bind/named.conf.options} før du setter i gang.
For å spare dere litt forvirring bør dere slå av rekursjon og få Bind til å lytte 
på en annen offentlig IP enn tinydns. Bruk:

\begin{filedata}
recursion: no;
listen-on \{ 158.38.74.ZYX; \};
listen-on-v6 \{ none; \};
\end{filedata}

Dette er omtalt nederst på side 253 i læreboka. 

En naturlig plassering for sonefilen for ditt domene er 
\file{/etc/bind/db.lpcXYZ.himolde.no}.  
Du legger til sonen i fila \file{/etc/bind/named.conf.local}. 
Du kan se i \file{/etc/bind/named.conf.default-zones} for 
å lære hvordan du gjør det. Hvis 

\cmd{service bind9 reload}

ikke virker som forventer, stopper og starter du bare tjenesten.

For å feilsøke Bind konfigurasjonen så har du to verktøy: \\ 
\prog{named-checkconf} og \prog{named-checkzone}.
\index{named-checkconf|@see{Bind}}\index{Bind!named-checkconf}
\index{named-checkzone|@see{Bind}}\index{Bind!named-checkzone}

For å sjekke selve konfigurasjonen (utenom sonefilene eller ressursdata om du vil), bruk:

\cmd{named-checkconf} 

For å sjekke en sonefil (og dens ressursdata), bruk:

\cmd{named-checkzone lpcXYZ.himolde.no /etc/bind/db.lpcXYZ.himolde.no}

% ######################################################
\section{Læreboka og den obligatoriske oppgaven}
% ######################################################

Når læreboka snakker om \textquote{z.example.com} (første gang på side 243),
så bytter du det ut med ditt domene - \textquote{lpcXXX.himolde.no}. Det er litt 
mer naturlig.

Som det står i oppgaveteksten skal Bind ha samme ressursdata som
tinydns. Du skal altså \textbf{ikke} delegere autoritet for et barnedomene fra
tinydns til bind - ref 8.5.3 i læreboka.

Hvis du lurer på hvilke tjenester (eller prosesser)
som lytter på port 53 (DNS), bruk

\cmd{netstat -anp | grep ':53 '}

