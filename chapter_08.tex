\chapter{Dovecot} % kapt 7

\section{Installasjon}

Lærebok bruker \prog{qmail-pop3d} og \prog{qpopper}. Dessverre kan 
ingen av disse brukes.

\begin{itemize}
\item \prog{qmail-pop3d} kan ikke brukes fordi programmet checkpassword ikke 
   virker. (Ref kap. 7.3)
\item \prog{qpopper} kan ikke brukes fordi det ikke lenger vedlikeholdes og 
   det er heller ikke tilgjengelig for Debian. (Ref kap. 7.5)
\end{itemize}

Det er mange andre muligheter (som jeg lister opp i forelesningen),
men jeg tror det er minst arbeid å prøve Dovecot sin POP3. 

\cmd{apt install dovecot-pop3d}

Endre konfigurasjonen for Dovecot slik at det fungerer uten noe problemer.
Opprett fila \file{/etc/dovecot/local.conf} med følgende to linjer:

\begin{filedata}
disable_plaintext_auth = no
mail_access_groups = mail
\end{filedata}

Få Dovecot til å lese konfigurasjonen på nytt. OK, hvis du lurer
så er det selvfølgelig
   
\cmd{service dovecot reload}

\section{Testing av POP3}

Jeg syns det er lurt at dere prøver POP3 på protokollnivå (med
telnet) slik som vi gjorde med HTTP og SMTP. Bruk

\cmd[\$]{telnet lpcXYZ.himolde.no pop3}

Denne kommandoen kan du kjøre som deg selv (uten \prog{sudo}).
Aller best er det hvis du gjør det fra en annen maskin 
(f.eks ibe101.himolde.no) slik at du er helt sikker på at det virker.

Prøv gjerne fra din egen PC også med f.eks. 
\href{https://www.mozilla.org/nb-NO/thunderbird/}{Thunderbird}

\begin{remark}
Siden vi kjører POP3 uten kryptering anbefaler jeg at du 
legger til en egen testbruker for POP (med \prog{adduser}).
\end{remark}