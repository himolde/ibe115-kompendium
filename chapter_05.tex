\chapter{Samba}\index{Samba} % kapt 3


\section{Installasjon}

Læreboka installerer Samba fra kildekode, ikke fra binærpakke eller
med pakkeprogram som jeg sier i dette kompendiet. Grunnen til dette 
var nok at Samba var i stor utvikling da boken ble skrevet, og det var
viktig å ha en så ny utgave som mulig. Vi skal altså se bortifra alle 
referanser til \file{/usr/local/samba}.

Som vanlig installerer man Samba med \prog{apt}

\cmd{apt install samba}

Jada, det funger fortsatt/også med \prog{apt-get} hvis du er vant til det.
(Jeg dropper denne setningen i neste kapittel.)

Hvis du trenger å teste tilkobling til Samba enkleste mulig, uten å rote 
med en annen maskin, så bruker du \prog{smbclient}. Du installerer med

\cmd{apt install smbclient}\index{Samba!smbclient}\index{smbclient|see{Samba}}


\section{Dokumentasjon}

Den mest oppdaterte dokumentasjonen for Samba finner
du alltid på \url{https://wiki.samba.org/} 

Hvis du vet omtrent hva du leter etter, så er det i mine øyne
enklere og bedre, å lese man-siden for \file{smb.conf}. 

Debian kommer med Samba 4 (versjon 4.5.8 per 12.09.2017), men heldigvis er ikke 
endringene så store fra Samba 3. Dermed kan du fortsatt lese
\href{https://www.samba.org/samba/docs/man/Samba3-HOWTO/}%
{The Official Samba 3.5.x HOWTO and Reference Guide}.


\section{Konfigurasjon}

Læreboka er stort sett korrekt for Samba, men noen få ting er det verdt å merke seg.

For det først skal dere ikke bruke \prog{/usr/local/samba/bin/smbpasswd} når dere 
skal sette passord for Samba-brukere. Kommandoen er \prog{smbpasswd} - uten absolutt 
sti som vanlig. \index{Samba!smbpasswd}\index{smbpasswd|see{Samba}}
Dere husker at vi ikke har installert fra kilde, ikke sant? 

Følgende små endringer i forhold i læreboka:

\begin{itemize}
    \item Ikke oppgi \config{smb passwd file}. Stien til passordfilen er allerede 
        korrekt kompilert inn i Samba. (Filen er \file{/var/lib/samba/private/passdb.tdb}.)
        Hvis dere ønsker å se brukerne som er i filen, bruk \cmd{pdbedit -L}.
        \index{Samba!pdbedit}\index{pdbedit|see{Samba}}
    \item Hvis dere vil koble til fra Windows, så må dere kanskje legge til
        \config{ntlm auth = yes}. Grunnen til dette er at Windows som standard ikke
        bruker NTLMv2 som er nyere/sikrere. (Istedenfor å aktivere NTML, så kan dere
        endre sikkerhetspolicy for deres PC, men det er mer komplisert.)
        Samba 4 har byttet til NTLMv2 som standard.
    \item Samba 4 har introdusert \config{server role}. Jeg anbefaler at dere bruker denne
        istedenfor \config{security}.
\end{itemize}

NB1! Husk å test/sjekke konfigurasjonen etter at du har endret den ved hjelp av \prog{testparm}.

NB2! Husk at konfigurasjonen må leses inn på nytt etter at du har endret \file{smb.conf}:

\cmd{service smbd reload}

PS! Det er lov å bruke 

\cmd{systemctl reload smbd.service}

men jeg kommer altså til å bruke \prog{service} som jeg skrev i kapittel 1.
