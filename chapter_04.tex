\chapter{Apache}\index{Apache}  % kapt 4

Dokumentasjonen for siste/gjeldende utgave av Apache finner
du alltid på\\ \url{http://httpd.apache.org/docs/current/}

I dette kapitlet vil vi vise til diverse informasjon for 
utgave 2.4 og dermed istedenfor lenke til\\
\url{http://httpd.apache.org/docs/2.4/} \cite{docs:apache}

\section{Installasjon}

Som vanlig installerer man (Apache) med \prog{apt}

\cmd{apt install apache2}

Jada, det funger fortsatt/også med \prog{apt-get} hvis du er vant til det.

Hvis du har lyst til å leke litt med PHP (og seinere installere Wordpress
eller et annet PHP-basert CMS), så kan du installere PHP-støtte i Apache med

\cmd{apt install libapache2-mod-php}\index{PHP}

Dette legger til en PHP-modul i Apache som er den vanligste måten å kjøre PHP på.

Et annet alternativ som er ganske vanlig hos en del hostingselskaper 
er FastCGI-modulen sammen med PHP-FPM.

\section{Konfigurasjon}

Som tidligere påpekt bruker læreboka Apache 2.2, mens vi skal bruke Apache 2.4. 
Dette er standardvalget i Debian og gjeldene utgave for Apache. Før vi tar
noen spesifikke detaljer, merk at:

\begin{remark}
Læreboka snakker om \file{httpd.conf}\index{httpd.conf} som Apache sin konfigurasjonsfil. 
Den samme fila heter \file{apache2.conf}\index{apache2.conf} i Debian og ligger selvfølgelig i 
mappa \file{/etc/apache2/}. Før du starter med denne oppgaven bør du lese starten 
av \file{apache2.conf} (på din server) - bruk less. Du vil da skjønne at 
fila \file{apache2.conf} er satt sammen av flere filer. Dette er istedenfor at 
all konfigurasjon er samlet i en \textquote{gigantisk} fil.
\end{remark}

\subsection{Overgang fra 2.2 til 2.4}

Dette er godt beskrevet i dokumentet
\href{http://httpd.apache.org/docs/2.4/upgrading.html}{Upgrading to 2.4 from 2.2}.
Du må lese om \textquote{Access control} som viser hvordan man setter opp
tilgangskontroll i Apache 2.4. Dette er altså en korreksjon til kapittel 4.8 
i lærebok. I svært korte trekk kan man si at \config{Order}, \config{Allow} og \config{Deny}
er byttet ut med \config{Require}\index{Apache!direktiver!Require}\index{Require|see{Apache, direktiver}}.

\subsection{Direktiver}

Apache blir styrt gjennom såkalte direktiver i konfigurasjonsfilene. Alle direktivene 
er godt dokumentert, med eksempler, i Apache sin dokumentasjon - se
\href{http://httpd.apache.org/docs/2.4/mod/quickreference.html}{Directive Quick Reference}.
Etterhvert som man vet omtrent hva direktivene heter, så trenger man ikke mer
enn denne oversikten. 

Den kanskje kraftigste funksjonaliteten til Apache finner man i modulen Rewrite.
\index{Apache!moduler!Rewrite}\index{Rewrite|see{Apache, moduler}}\index{mod\_rewrite|see{Apache, moduler}}
Denne blir ikke omtalt i læreboka, men dere kan lese om dette i håndboken
\href{http://httpd.apache.org/docs/2.4/rewrite/}{URL Rewriting with mod\_rewrite}.

En annen nyttig modul er UserDir som lar deg sette opp hjemmesider for hver enkelt
bruker på serverene. Dere kan lese mer om UserDir i howto-en
\href{http://httpd.apache.org/docs/2.4/howto/public\_html.html}{Per-user web directories}.
\index{Apache!direktiver!UserDir}\index{Apache!moduler!UserDir}
\index{UserDir|see{Apache, Moduler}}\index{mod\_userdir|see{Apache, moduler}}

\section{Trafikkanalyse}\index{trafikkanalyse}\index{logganalyse}

Når man driver et nettsted er det viktig å vite så mye som mulig om de besøkende.
Tidligere ble dette gjort ved å se på loggfilene til webserveren (Apache). Kjente 
programmer for dette er Webalizer og AWStats.\index{Webalizer}\index{AWStats}

Moderne analyseverktøy baserer seg på sporing på klientsiden (i nettleseren) ved
hjelp av JavaScript. Den meste kjente løsningen er Google Analytics. Hvis du 
ønsker å ha fullkontroll over dataene dine eller lovgivning sier at du ikke kan bruke 
en tredjepart til å samle inn data, så er løsningen Piwik.\index{Piwik}
Jeg anbefaler at dere tester det ut hvis dere får tid. 
(Piwik fins for tiden dessverre ikke som en offisiell Debian-pakke.)

