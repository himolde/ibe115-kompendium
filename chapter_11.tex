\chapter{Nextcloud / Filsynkronisering}\index{Nextcloud}
\index{synkronisering|see{filsynkronisering}}
\index{filsynkronisering}
\index{backup}

Før vi installerer programvare for filsynkronisering så er det viktig
å ha klart for oss skillet mellom backup og synkronisering.

Filsynkronisering holder dine forskjellige enheter oppdatert slik at
samme innhold (filer) er tilgjengelig på alle enhetene. Dette har den
åpenbare fordelen at man f.eks. kan fortsette å arbeide hjemme etter 
arbeid eller lett se på filene på mobilen. I forhold til å jobbe mot
en nettverksdisk, så er filsynkronisering overlegent når det kommer
til ytelse siden du jobber lokalt på enheten, og filene bare 
synkroniseres (lastes opp) når det oppstår endringer. Ulempen med 
synkronisering er at det av og til oppstår konflikter fordi du 
jobber med filene på flere enhet samtidig. 

Mange bruker filsynkronisering som backup, siden filene faktisk er 
trygt lastet opp til skyen/en server. Det er to problemer med dette.
For det første så er det ingen lett måte å hente tilbake alle filene
fra et gitt tidspunkt. Filsynkronisering tilbyr som oftes 
versjonskontroll (slik at du kan gå tilbake), men denne er per fil og
kan være begrenset i tid eller antall. For det andre så er det altfor
lett at \textquote{backupen} blir overskrevet ved f.eks. virusangrep
fordi synkroniseringen pågår kontinuerlig.

Backup skiller seg fra filsynkronisering ved at man tar en kopi av hele
disken eller hele mapper på gitte tidspunkt (en gang om dagen f.eks.) 
og lagrer disse i skyen. Dette gjøres typisk fra hovedenhet, og 
backupen bør som regel inneholde mappene som du synkroniserer 
med de andre enhetene. Backup bør også tilbyr muligheten til å gå 
tilbake i tid, ikke bare til siste backup, for å gjenskape (restore) 
hele disken slik den var tidligere.

De store kommersielle tilbyderne av filsynkronisering er Dropbox,
iCloud Drive, OneDrive og Google Drive (som nå også tilbyr backup). 
Alle disse har begrensede utgaver som de tilbyr gratis til privatpersoner.

Det kan være mange grunner til at man ikke ønsker å benytte seg av 
disse kommersielle aktørene - regelverk, personvern og ønske om full
kontroll med egne data. Heldigvis kan man sette opp slike tjenester
selv. Man vil kanskje ikke få samme ytelse som hos de store aktørene,
men man har i alle fall full kontroll. Vi skal prøve en av åpen 
kildekode alternativene, \href{https://nextcloud.com/}{Nextcloud} 
(som er en variant av \href{https://owncloud.org/}{Owncloud}).   

\section{Installasjon}

Hverken Nextcloud eller Owncloud er tilgjengelig via Debian sitt 
pakkesystem. Såvidt jeg vet er det fordi det er \textquote{umulig}
for Debian-pakkevedlikeholderne å tilby vanlig oppgradering 
mellom versjonene slik det er forventet. Dermed må vi 
\href{https://docs.nextcloud.com/server/12/admin_manual/installation/source_installation.html}{installere manuelt}, 
men vi skal ikke kompilere noe. Vi kunne installert
Nextcloud som en Snap-pakke, men jeg tenker at vi holder oss til
det vanlige pakkesystemet. Nextcloud krever en såkalt LAMP-stack - 
Linux, Apache, MySQL/MariaDB og PHP. Vi mangler bare to bokstaver ...

\subsection*{Forhåndskrav}

Nextcloud krever masse PHP-moduler, men følgende kommandoer dekker det meste:

\cmd{apt install mariadb-server libapache2-mod-php7.0}\\
\cmd{apt install php7.0-gd php7.0-json php7.0-mysql php7.0-curl}\\
\cmd{apt install php7.0-mbstring php7.0-intl}\\
\cmd{apt install php7.0-mcrypt php-imagick php7.0-xml php7.0-zip}

\subsection*{Nedlasting}

Siste utgave akkurat nå er 12.0.3. Hvis det kommer nye utgaver, oppdaterer du bare
versjonnummeret i kommandoene nedenfor.

\cmd{wget https://download.nextcloud.com/server/releases/nextcloud-12.0.3.tar.bz2}\\
\cmd{tar xf nextcloud-12.0.3.tar.bz2}\\
\cmd{mv nextcloud /var/www}

\section{Konfigurasjon}

\subsection*{Apache}

Opprett \file{/etc/apache2/sites-available/nextcloud.conf} med følgende innhold:

\begin{filedata}
Alias /nextcloud "/var/www/nextcloud/"

<Directory /var/www/nextcloud/>
  Options +FollowSymlinks
  AllowOverride All

 <IfModule mod_dav.c>
  Dav off
 </IfModule>

 SetEnv HOME /var/www/nextcloud
 SetEnv HTTP_HOME /var/www/nextcloud

</Directory>
\end{filedata}

Aktiver disse innstillingene og diverse Apache-moduler:

\cmd{a2ensite nextcloud}\\
\cmd{a2enmod rewrite headers env dir mime}

Du må nesten aktivere kryptering også:

\cmd{a2enmod ssl}\\
\cmd{a2ensite default-ssl}

\subsection*{MariaDB/MySQL}

Du må opprette en database og en databasebruker for Nextcloud.

Lag en fil \file{create\_nextcloud.sql} som inneholder

\begin{filedata}
CREATE USER 'username'@'localhost' IDENTIFIED BY 'password';
CREATE DATABASE IF NOT EXISTS nextcloud;
GRANT ALL PRIVILEGES ON nextcloud.* TO 
  'username'@'localhost' IDENTIFIED BY 'password';
\end{filedata}

Du må gjerne endre \textquote{username} og \textquote{password}. 
Husk disse og at databasen heter \textquote{nextcloud}.

Utfør selve SQL-kommandoene:

\cmd{mysql < create\_nextcloud.sql}

\subsection*{Nextcloud}

Gjør følgende endringer på filsystemet:

\cmd{cd /var/www/nextcloud}\\
\cmd{mkdir data}\\
\cmd{chown -R www-data apps config data}

Du er nå klar til å fullføre konfigurasjon/installasjon av Nextcloud. Gå til

\url{https://lpc1XY.himolde.no/nextcloud/}

Etter at installasjonen er ferdig, så er du automatisk innlogget som admin. 
Du bør opprette en vanlig bruker for deg selv, direktelenke:

\url{https://lpc1XY.himolde.no/nextcloud/index.php/settings/users}

\subsection*{Fine URL-er}

Hvis du vil har URL-er uten \textquote{index.php}, så må du legge til

\begin{filedata}
  'htaccess.RewriteBase' => '/nextcloud',
\end{filedata}

etter overwrite.cli.url-linja i \file{/var/www/nextcloud/config/config.php}.

Deretter må du oppdatere filen \file{/var/www/nextcloud/.htaccess}. Det gjør du lettest slik:

\cmd{cd /var/www/nextcloud/}\\
\cmd{chown www-data .htaccess}\\
\cmd{sudo -u www-data php occ maintenance:update:htaccess} 

På grunn av en feil vises ikke lenger ikonene oppe til høyre etter at du har aktiveert
fine URL-er. For å få dem frem igjen, må du gå til

\url{https://lpc1XY.himolde.no/nextcloud/settings/admin/theming}

og f.eks. endre navnet. Da skal ikonene vises igjen. Du kan endre navn tilbake med en gang.

\section{Bruk}

Nextcloud har klienter for PC/Desktop og mobiltelefoner. Jeg har bare testet Andorid-appen
som er gratis - og fungerer. Filene blir synkronisert mellom telefonen og nettstedet. Jeg 
burde selvfølgelig testet på PC-en også, men nå har jeg brukt nok tid ...