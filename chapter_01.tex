\chapter{Linux}\index{Linux} % kommandolinja og sentrale verktøy

Linux er kjernen til operativsystemet som vanligvis går under samme navn. Skal man være
helt korrekt bør operativsystemet kalles \textbf{GNU/Linux}\index{Linux!GNU/Linux} 
fordi userland kommer fra \href{https://www.gnu.org/}{GNU-prosjektet}
som er drevet av \href{https://www.fsf.org/}{Free Software Foundation (FSF)}.

Det viktigste å huske er at hele Linux-operativsystemet (eller GNU/Linux om du vil) er 
\href{https://en.wikipedia.org/wiki/Free_and_open-source_software}{fri og åpen programvare (FOSS)}.
Husk at fri og åpen programvare ikke bare betyr at det er gratis, som jo er en stor fordel, men 
også at du kan endre og tilpasse programvaren. Dette gjør at Linux har fått en voldsom utbredelse
og brukes i alt fra ordinære servere og nettverksutstyr, 
\href{https://en.wikipedia.org/wiki/Internet_of_things}{IOT}-utstyr, mobiltelefoner, bilen din, 
kjøleskapet ditt \dots Linux fungerer også flott som vanlig PC, men det er ikke den vanligste 
bruken.

GNU/Linux er svært moden programvare. Utviklingen av GNU startet allerede 1983, og første utgave 
av Linux ble sluppet i 1991 av finnen Linus Torvalds som fortsatt styrer utvikling av Linux.
Vil du vite mer om Linux, så er 
\href{https://en.wikipedia.org/wiki/Linux}{Linux-artikkelen på Wikipedia} \cite{wiki:linux}
en god start.

\section{Administratortilgang}\index{root}\index{sudo}

For å kunne administrere serveren din i dette emnet (og eventuelt seinere i arbeidslivet) 
trenger du administratortilgang. Superbrukeren på en Linux-server heter \file{root}, og du
trenger denne kontoen sitt passord for å kunne logge inn. Dette er en stor ulempe hvis flere
skal administrere serveren - du vet ikke hvem som logget inn og du kan heller ikke endre 
\file{root}-passordet utenat det påvirker de andre.

Den vanligste løsning på dette problemet er å bruke \prog{sudo} - \textquote{super user do}. 
Hvis du får sudo-tilgang, så kan du utføre kommandoer som om du var \file{root} uten at du
trenger å logge inn som \file{root}. \prog{sudo} spør om ditt passord - og husker det en stund
slik at du slipper å skrive passordet hele tiden. En ekstra fordel med \prog{sudo} er at
du kan gi en bruker bare delvis administratortilgang. 

Hvis du trenger å være \file{root} over en lengre periode, så kan du bruke 

\cmd[\$]{sudo su -} 

til å bli brukeren \file{root}. Du hopper tilbake til din egen bruker med \prog{exit}.
Faren ved å være \file{root} hele tiden, er at du kan gjøre stor skade uten at du er klar over
det. Skriver du alltid sudo så blir du litt oppmerksom på at du har superkrefter.

\begin{remark}
Jeg har droppet sudo fra kommandolinje-eksemplene i dette kompendiet. Jeg bruker altså

\cmd{service apache2 reload}

istedenfor

\cmd[\$]{sudo service apache2 reload}

Dette er selvfølgelig bare latskap. Legg også merke til at kommandolinja starter 
med \# når jeg er \file{root} og \$ når jeg er en ordinære bruker. 
Du kan sjekke hvem du er med \prog{whoami} ...
\end{remark}

