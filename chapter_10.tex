\chapter{ProFTP} % kapt 10
\index{ProFTP}\index{ftp}

\section{Installasjon}

\cmd{apt install proftpd}

Du vil gjerne også ha en FTP-klient slik at du kan teste lokalt:

\cmd{aptitude install ftp}

\section{Konfigurasjon}

ProFTP virker ut av boksen. (Tjenesten heter proftpd.) 
For å teste at det fungerer, bruk

\cmd[\$]{ftp localhost}

og logg inn som din stud-bruker. Hvis du vil teste utenfra, så kan du bruke
FileZilla på din egen laptop. Siden FTP ikke er kryptert, så kan det lønne
seg å opprette en testbruker, f.eks. \file{ftpbruker}, slik at ingen 
plukker opp passordet ditt. (Dette er ikke et problem hvis du tester lokalt 
mot localhost.)

\index{jail}\index{chroot|see{jail}}
Det er svært vanlig å \textquote{jaile} brukeren så han ikke kommer utenfor
sitt eget hjemmeområde. Dette kalles også ofte for \textquote{chroot}. 

Sammenlign resultatet av pwd (i \prog{ftp}) før og etter du har aktivert 
\config{DefaultRoot} i konfigurasjonsfilen \file{/etc/proftpd/proftpd.conf}. 
Husk å laste konfigurasjonen til proftpd på nytt. 
Jeg regner med at dere skjønner at dette gir et ekstra lag av sikkerhet.

Dokumentasjon for alle direktivene finner dere på\\
\url{http://www.proftpd.org/docs/directives/linked/by-name.html}